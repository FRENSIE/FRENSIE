\documentclass{beamer}
\usetheme[white]{Wisconsin}
\usepackage{wrapfig}
\usepackage{longtable}
\usepackage{listings}
\usepackage{color}
%% The amssymb package provides various useful mathematical symbols
\usepackage{amssymb}
%% The amsthm package provides extended theorem environments
\usepackage{amsthm} \usepackage{amsmath}
\usepackage[mathcal]{euscript} \usepackage{color}
\usepackage{textcomp}
\usepackage{algorithm,algorithmic}
\usepackage[absolute,overlay]{textpos}
  \setlength{\TPHorizModule}{1mm}
  \setlength{\TPVertModule}{1mm}
\definecolor{listinggray}{gray}{0.9}
\definecolor{lbcolor}{rgb}{0.9,0.9,0.9}
\lstset{
  backgroundcolor=\color{lbcolor},
  tabsize=4,
  rulecolor=,
  language=c++,
  basicstyle=\scriptsize,
  upquote=true,
  aboveskip={1.5\baselineskip},
  columns=fixed,
  showstringspaces=false,
  extendedchars=true,
  breaklines=true,
  prebreak =
  \raisebox{0ex}[0ex][0ex]{\ensuremath{\hookleftarrow}},
  frame=single,
  showtabs=false,
  showspaces=false,
  showstringspaces=false,
  identifierstyle=\ttfamily,
  keywordstyle=\color[rgb]{0,0,1},
  commentstyle=\color[rgb]{0.133,0.545,0.133},
  stringstyle=\color[rgb]{0.627,0.126,0.941},
}

%% colors
\setbeamercolor{boxheadcolor}{fg=white,bg=UWRed}
\setbeamercolor{boxbodycolor}{fg=black,bg=white}

\setbeamerfont{block body}{size=\footnotesize}

%%----------------------------------------------------------------------------%%
\author{Alex P. Robinson
    \\ Engineering Physics Department
    \\ University of Wisconsin - Madison
    \\ Preliminary Examination
}

\date{\today}
\title{Development of a Monte Carlo Code System with Continuous Energy Adjoint Transport Capabilities for Neutrons and Photons} 
\begin{document}
\maketitle
%%----------------------------------------------------------------------------%%
\section{Overview}
%%----------------------------------------------------------------------------%%
\begin{frame}{Overview}

\begin{itemize}
  \item The Monte Carlo method
  \item Motivations for using the adjoint process
  \item Monte Carlo codes available today
  \item The Monte Carlo random walk process
  \item The Monte Carlo random walk process for radiation transport
  \item The Monte Carlo random walk process for adjoint radiation transport
  \item Adjoint photon incoherent scattering
  \item Adjoint photon coherent scattering
  \item Adjoint photon pair production
  \item The adjoint photon weight factor
  \item FACEMC code overview
  \item FACEMC validation plan
  \item Future Work
\end{itemize}

\end{frame}

%%----------------------------------------------------------------------------%%
\section{The Monte Carlo Method}
%%----------------------------------------------------------------------------%%
\begin{frame}{The Monte Carlo method}

\end{frame}

%%----------------------------------------------------------------------------%%
\begin{frame}{The forward process vs. the adjoint process}

\end{frame}

%%----------------------------------------------------------------------------%%
\section{Motivations for using the adjoint process}
%%----------------------------------------------------------------------------%%
\begin{frame}{Motivations for using the adjoint process}

\begin{itemize}
  \item Two motivations for using the adjoint process will be given:
    \begin{enumerate}
      \item One based on the phase space of the problem
      \item One based on the physical interpretation of the quantity that is 
        estimated during the adjoint process
    \end{enumerate}
\end{itemize}

\end{frame}

%%----------------------------------------------------------------------------%%
\begin{frame}{1.) A phase space motivation for using the adjoint process}

\end{frame}

%%----------------------------------------------------------------------------%%
\begin{frame}{2.) A motivation based on the physical interpretation of the adjoint flux}

\end{frame}

%%----------------------------------------------------------------------------%%
\section{Monte Carlo Codes Available Today}
%%----------------------------------------------------------------------------%%
\begin{frame}{Continuous energy capabilities of popular codes}

\end{frame}

%%----------------------------------------------------------------------------%%
\section{The Monte Carlo random walk process}
%%----------------------------------------------------------------------------%%
\begin{frame}{Fredholm Integral Equations of the second kind (FIESKs)}

  \begin{itemize}
    \item Volterra integral equation also important for radiation transport.
    \item It is a special case of the FIESK
   \end{itemize}

\end{frame}

%%----------------------------------------------------------------------------%%
\begin{frame}{Analytical solution method: the method of successive approximations}

\end{frame}

%%----------------------------------------------------------------------------%%
\begin{frame}{Numerical solution method: the Monte Carlo random walk process}

\end{frame}

%%----------------------------------------------------------------------------%%
\begin{frame}{Proof that the Monte Carlo method recovers the solution}

\end{frame}

%%----------------------------------------------------------------------------%%
\section{The Monte Carlo Random Walk Process for Radiation Transport}
%%----------------------------------------------------------------------------%%
\begin{frame}{The Boltzmann or transport equation}

  \begin{itemize}
    \item The transport equation described the expected behavior of particles
      in a medium.
    \item This equation must be converted to a FIESK before a Monte Carlo
      random walk process can be employed.
  \end{itemize}

\end{frame}

%%----------------------------------------------------------------------------%%
\begin{frame}{Converting the transport equation to an integral form}

\end{frame}

%%----------------------------------------------------------------------------%%
\begin{frame}{The emission density FIESK}

\end{frame}

%%----------------------------------------------------------------------------%%
\begin{frame}{The collision density FIESK}

\end{frame}

%%----------------------------------------------------------------------------%%
\begin{frame}{The transport kernel}

\end{frame}

%%----------------------------------------------------------------------------%%
\begin{frame}{The collision kernel}

\end{frame}

%%----------------------------------------------------------------------------%%
\begin{frame}{The Monte Carlo random walk process for radiation transport}

\end{frame}

%%----------------------------------------------------------------------------%%
\section{The Monte Carlo random walk process for adjoint radiation transport}
%%----------------------------------------------------------------------------%%
\begin{frame}{Derivation of the adjoint transport equation}

\end{frame}

%%----------------------------------------------------------------------------%%
\begin{frame}{Converting the adjoint transport equation to an integral form}

\end{frame}

%%----------------------------------------------------------------------------%%
\begin{frame}{The adjoint emission density}

\end{frame}

%%----------------------------------------------------------------------------%%
\begin{frame}{The adjoint collision density}

\end{frame}

%%----------------------------------------------------------------------------%%
\begin{frame}{The adjoint transport kernel}

\end{frame}

%%----------------------------------------------------------------------------%%
\begin{frame}{The adjoint collision kernel}

\end{frame}

%%----------------------------------------------------------------------------%%
\begin{frame}{Adjoint cross sections}

\end{frame}

%%----------------------------------------------------------------------------%%
\begin{frame}{The adjoint weight factor}

\end{frame}

%%----------------------------------------------------------------------------%%
\begin{frame}{The Monte Carlo random walk process for adjoint radiation transport}

\end{frame}

%%----------------------------------------------------------------------------%%
\section{Adjoint Photon Incoherent Scattering}
%%----------------------------------------------------------------------------%%
\begin{frame}{The incoherent scattering cross section for photons}

\end{frame}

%%----------------------------------------------------------------------------%%
\begin{frame}{Developing the adjoint incoherent scattering cross section}

\end{frame}

%%----------------------------------------------------------------------------%%
\begin{frame}{The adjoint incoherent scattering cross section for photons}

\end{frame}

%%----------------------------------------------------------------------------%%
\begin{frame}{The adjoint incoherent scattering cross section for aluminum}

\end{frame}

%%----------------------------------------------------------------------------%%
\section{Adjoint Photon Coherent Scattering}
%%----------------------------------------------------------------------------%%
\begin{frame}{The coherent scattering cross section for photons}

\end{frame}

%%----------------------------------------------------------------------------%%
\begin{frame}{Developing the adjoint coherent scattering cross section}

\end{frame}

%%----------------------------------------------------------------------------%%
\section{Adjoint Photon Pair Production}
%%----------------------------------------------------------------------------%%
\begin{frame}{The simplified pair production cross section for photons}

\end{frame}

%%----------------------------------------------------------------------------%%
\begin{frame}{Developing the adjoint pair production cross section}

\end{frame}

%%----------------------------------------------------------------------------%%
\begin{frame}{The adjoint pair production kernel}

\end{frame}

%%----------------------------------------------------------------------------%%
\begin{frame}{A modified Monte Carlo random walk process for adjoint photon transport}

\end{frame}

%%----------------------------------------------------------------------------%%
\section{The Adjoint Photon Weight Factor }
%%----------------------------------------------------------------------------%%
\begin{frame}{The adjoint photon weight factor}

\end{frame}

%%----------------------------------------------------------------------------%%
\section{FACEMC Code Overview}
%%----------------------------------------------------------------------------%%
\begin{frame}{FACEMC code requirements}

\end{frame}

%%----------------------------------------------------------------------------%%
\begin{frame}{Major Components of FACEMC}

\end{frame}

%%----------------------------------------------------------------------------%%
\section{FACEMC Validation Plan}
%%----------------------------------------------------------------------------%%
\begin{frame}{FACEMC validation plan}

\end{frame}

%%----------------------------------------------------------------------------%%
\begin{frame}{FACEMC validation plan step 1}

\end{frame}

%%----------------------------------------------------------------------------%%
\begin{frame}{FACEMC validation plan step 2}

\end{frame}

%%----------------------------------------------------------------------------%%
\begin{frame}{FACEMC validation plan step 3}

\end{frame}

%%----------------------------------------------------------------------------%%
\begin{frame}{Results from preliminary validation of FACEMC}

\end{frame}

%%----------------------------------------------------------------------------%%
\section{Future Work}
%%----------------------------------------------------------------------------%%
\begin{frame}{Future work on FACEMC}

\end{frame}

%%----------------------------------------------------------------------------%%

\end{document}

