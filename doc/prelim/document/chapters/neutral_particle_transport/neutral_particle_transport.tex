\chapter{Monte Carlo Methods for Neutral Particle Transport}
\label{ch:neutral_particle_transport}
In any Monte Carlo process, information collected from samples of the parent 
population will only be useful if the probability laws governing the sampling
are an accurate represention of a physical process. In this chapter, the 
probability laws governing the sampling of neutral particle histories (i.e. 
neutrons and photons) will be derived for the forward process. After a 
discussion of the various probability laws is given, the probability laws
governing the sampling of neutral particle histories will be derived for the
reverse process. In addition, the methods used to calculate the adjoint
cross section data will be discussed.

\section{Derivation of Probability Laws for the Forward Process}
\label{sec:der_prob_laws_for_proc}
A natural starting point for deriving the probability laws for the forward 
radiation transport process is the time-independent Boltzmann equation. 
The Boltzmann equation, 
though technically a balance equation, characterizes the transport of neutral 
radiation (i.e. neutrons and photons) through a medium. The integro-differential
form of the equation is the following:
\begin{equation}
  \begin{split}
    \hat{\Omega} \cdot \vec{\bigtriangledown} \varphi(\vec{r},E,\hat{\Omega})
    & + \Sigma_T(\vec{r},E) \varphi(\vec{r},E,\hat{\Omega}) = 
    S(\vec{r},E,\hat{\Omega}) + \\
    & \int\int \Sigma_S(\vec{r},E^{'} \to E,\hat{\Omega^{'}} \to \hat{\Omega})
  \varphi(\vec{r},E,\hat{\Omega^{'}}) dE^{'}d\hat{\Omega^{'}}
  \end{split}
  \label{eq:integro_diff_boltzman_eqn}
\end{equation}
In equation \ref{eq:integro_diff_boltzman_eqn} $\varphi(\vec{r},E,\hat{\Omega})$
is the flux of particles at position $\vec{r}$ with energy $E$ and direction
$\hat{\Omega}$. $S(\vec{r},E,\hat{\Omega})$ is the source density. The first
step in gleaning information about the probability laws from this equation is
to convert it into its integral form. 

The streaming term in equation \ref{eq:integro_diff_boltzman_eqn} can be 
rewritten in terms of a directional derivative. Assume that $\vec{r}$ is a 
fixed point. If a new variable $R$ is introduced, a
new point $\vec{r^{'}}$ relative to the fixed point can be constructed: 
$\vec{r^{'}} = \vec{r} - R\hat{\Omega}$. The variable $R$ is sometimes refered
to as a characteristic in the literature (ref). The derivative of the flux
with respect to this characteristic must be determined.
\begin{eqnarray*}
  \frac{d\varphi(\vec{r^{'}},E,\hat{\Omega})}{dR} & = & \frac{dx^{'}}{dR}
  \frac{\partial\varphi}{\partial x^{'}} + \frac{dy^{'}}{dR} 
  \frac{\partial\varphi}{\partial y^{'}} + \frac{dz^{'}}{dR}
  \frac{\partial\varphi}{\partial z^{'}} \nonumber \\
  & = & -\Omega_{x^{'}}\frac{\partial\varphi}{\partial x^{'}} -
  \Omega_{y^{'}}\frac{\partial\varphi}{\partial y^{'}} -
  \Omega_{z^{'}}\frac{\partial\varphi}{\partial z^{'}} \nonumber \\
  & = & -\hat{\Omega} \cdot 
  \vec{\bigtriangledown}\varphi(\vec{r},E) \nonumber
\end{eqnarray*}
Finally, this new equation for the streaming term must be subsituted back into
equation \ref{eq:integro_diff_boltzman_eqn} and the resulting equation must be
multiplied by the integrating factor $exp\left[\beta(\vec{r},R,E)\right]$. 
The exponent of the integrating factor is often refered to as the optical 
distance, which is the distance in the number of mean free paths (ref). 
Equation \ref{eq:optical_distance} gives the optical distance in terms of the 
total macroscopic cross section.
\begin{equation}
  \beta(\vec{r},R,E) = \int_0^R \Sigma_T(\vec{r}-R^{'}\hat{\Omega},E)dR^{'}
  \label{eq:optical_distance}
\end{equation}
\begin{equation*}
  \begin{split}
    &\left[-\frac{d\varphi(\vec{r^{'}},E,\hat{\Omega})}{dR}
    + \Sigma_T(\vec{r^{'}},E)\varphi(\vec{r^{'}},E,\hat{\Omega}) \right]
    exp\left[\beta(\vec{r},R,E)\right] \\
    & \quad = \left[S(\vec{r^{'}},E,\hat{\Omega}) + \int\int 
      \Sigma_S(\vec{r^{'}},E^{'} \to E,\hat{\Omega^{'}} \to \hat{\Omega})
      \varphi(\vec{r^{'}},E,\hat{\Omega^{'}}) dE^{'}d\hat{\Omega^{'}}\right] \\
    & \quad\quad\quad
    exp\left[\beta(\vec{r},R,E)\right] 
  \end{split}
\end{equation*}
\begin{equation*}
  \begin{split}
    &-\frac{d}{dR}\left[\varphi(\vec{r^{'}},E,\hat{\Omega})
    exp\left[\beta(\vec{r},R,E)\right]\right]\\
    & \quad = \left[S(\vec{r^{'}},E,\hat{\Omega}) + \int\int 
      \Sigma_S(\vec{r^{'}},E^{'} \to E,\hat{\Omega^{'}} \to \hat{\Omega})
      \varphi(\vec{r^{'}},E,\hat{\Omega^{'}}) dE^{'}d\hat{\Omega^{'}}\right] \\
    & \quad\quad\quad
    exp\left[\beta(\vec{r},R,E)\right] 
  \end{split}
\end{equation*}
By integrating the above equation with respect to $R$ from $0$ to $\infty$, the
integral form of the Boltzmann equation for the particle flux can be obtained. 
It is assumed that the flux goes to zero at $\infty$. 
\begin{equation}
  \begin{split}
    \varphi(\vec{r},&E,\hat{\Omega}) = \int_0^{\infty} 
    exp\left[\beta(\vec{r},R,E)\right]
    \Big[S(\vec{r}-R\hat{\Omega},E,\hat{\Omega}) + \\
    & \int\int 
    \Sigma_S(\vec{r}-R\hat{\Omega},E^{'} \to E,\hat{\Omega^{'}} \to \hat{\Omega})
    \varphi(\vec{r}-R\hat{\Omega},E,\hat{\Omega^{'}}) 
    dE^{'}d\hat{\Omega^{'}}\Big] dR
  \end{split}
  \label{eq:integral_boltzmann_eqn}
\end{equation}
