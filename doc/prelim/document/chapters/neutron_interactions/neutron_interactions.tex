\chapter{Neutron Interaction Cross Sections and Sampling Techniques}
\label{ch:neutron_ineractions}
To conduct a Monte Carlo random walk for neutrons the individual reactions that
make up the collision kernel must be discussed. In addition, the methods used
to sample from the differential interaction cross sections must be discussed.
Because of the increased complexity of reactions between a neutron and an a
atomic nucleus compared to photon-atomic reactions, closed form differential
interaction cross sections are quite rare. The models and procedures that are
outlined in the ENDF manual will therefore be relied upon heavily. In this
chapter all secondary particles other than neutrons will be neglected from the
cross sections and sampling procedures. 

\section{Elastic and Inelastic Level Scatering}
Elastic and inelastic level scattering are the two simplest scattering reactions
between a neutron and an atomic nucleus. In elastic scattering the internal 
state of the atomic nucleus is left unchanged. In inelastic level scattering 
the atomic nucleus is excited to a particular state, which has an energy that 
will be denoted as Q above the ground state. Elastic scattering can therefore 
be regarded as a special case of inelastic scattering with Q set equal to zero.
Because of the complicated nature of the interaction between a neutron and the
atomic nucleus it is difficult to develop an equation for the differential
elastic or inelastic level scattering cross section for all incoming neutron 
energies. The solution to this problem is to develop tabulated angular 
distributions for a range of incoming neutron energies, which can be found in 
the ENDF/B-VII.1 library \citep{chadwick_endf/b-vii.1_2011}. 

According to the ENDF/B-VII.1 manual, the angular distrubitions for elastic and 
inelastic level scattering are expressed as normalized probability 
distributions.
\begin{equation}
  p(\mu,E^{'}) = \frac{2\pi}{\sigma(E^{'})}\sigma(\mu,E^{'})
\end{equation}
\begin{equation}
  \int_{-1}^1p(\mu,E^{'})d\mu=1
\end{equation}
The angular distribition will be given in either a tabular form or as a 
Legendre polynomial series, which is shown below.
\begin{equation}
  p(\mu,E^{'}) = \sum_{l=0}^N\frac{2l+1}{2}a_l(E^{'})P_l(\mu)
\end{equation}
In addition, the outgoing angle cosine can be given in either the lab frame or
the center-of-mass (CM) frame. However, for two-body reactions like elastic and 
inelastic level scattering the scattering angle cosine will usually be given in 
the CM frame \citep{chadwick_endf/b-vii.1_2011}. 

Sampling an outgoing angle from the PDF given in the ENDF library can be done 
in many ways \citep{lux_monte_1991}. In older Monte Carlo codes, including older
versions of MCNP, the most common sampling technique was that proposed by 
Carter and Cashwell in which n equally probable intervals of the CDF are 
tabulated \citep{l._l._particle-transport_1975}. To select an outgoing angle 
cosine, one randomly selects an interval of the CDF and then selects the 
scattering angle cosine from a uniform PDF between the lower and upper 
boundaries of the selected interval \citep{lux_monte_1991}. This method is very 
computationally efficient but can become inaccurate when the PDF is given as a 
Legendre polynomial with many higher order terms 
\citep{x-5_monte_carlo_team_mcnp_2003}. The next best option is to use a 2-D 
tabular selection method \citep{x-5_monte_carlo_team_mcnp_2003}. This is very
similar to the tabular selection method which was discussed in the previous 
chapter. At each of the selected incoming neutron energies the CDF corresponding
to the PDF given in the ENDF library must first be calculated. To sample an 
outgoing angle cosine corresponding to a particular incoming neutron energy one
must determine which energy bin the neutron's energy falls in, which can be 
accomplished with a binary search. Then one samples a random number and 
for the distribution corresponding to both energy bin boundaries, determines 
the outgoing angle cosine corresponding to that CDF value. The outgoing angle
cosine corresponding to the incoming energy must then be found using
interpolation between the two values. A set of appropriate interpolation schemes
for 2-D tables are given in the ENDF/B-VII.1 manual 
\citep{chadwick_endf/b-vii.1_2011}. 

Once the CM scattering angle cosine is sampled, it must be converted to the 
lab scattering angle cosine. This equation, which is shown below, can be derived
using conservation of energy and momentum in both reference frames (see 
Appendix \ref{ch:appendix_C}).
\begin{equation}
  \mu_{l} = \frac{A\sqrt{1-\left(\frac{A+1}{A}\right)\frac{Q}{E^{'}}}\mu_{cm} + 1}
  {\sqrt{2A\sqrt{1-\left(\frac{A+1}{A}\right)\frac{Q}{E^{'}}}\mu_{cm} + 1 + A^2
        \left[1-\left(\frac{A+1}{A}\right)\frac{Q}{E^{'}}\right]}}
\end{equation}
\begin{equation*}
  A = \frac{m_A}{m_n}
\end{equation*}

Because of the one-to-one correspondence between the outgoing energy and
outgoing direction in the two-body reactions, once the outgoing angle cosine
has been sampled the outgoing energy of the neutron can be determined. The
equations that described this one-to-one correspondence can be found using
conservation of energy and momentum and the assumption that the atomic nucleus
upon which the neutron scatters is at rest. The outgoing energy from an 
inelastic level scattering interaction (in the lab frame) as a function of the 
initial energy and the scattering angle in the center of mass system is the 
following (see Appendix \ref{ch:appendix_C}).
\begin{equation}
  E = E^{'} \left(\frac{2A\sqrt{1 - \left(\frac{A+1}{A}\right)\frac{Q}{E^{'}}}
    \mu_{cm} + 1 + A^2\left[1-\left(\frac{A+1}{A}\right)\frac{Q}{E^{'}}
      \right]}{\left(A+1\right)^2}\right) 
\end{equation}

\section{Absorption Reactions}
A neutron absorption reaction is any reaction in which an incident neutron is
absorbed and another particle, whether a gamma ray, proton, alpha particle or
some other atomic nucleus, is emitted. When neutrons are the only particle of
interest, these reactions will all be combined into a single absorption
reaction. However, in coupled particle transport calculations, these reactions
along with the neutron collision density will function as the source term for
the other particles of interest. Some of these reactions will be discussed 
further in the following chapter when coupled neutron-photon transport will
be discussed. 

\section{Other Non-fission Reactions}
In the ENDF/B-VII.1 manual one can find a complete list of the neutron reactions
that are possible. A few examples are the inelastic scattering to 
a continuum of energy levels, the (n,2n) reaction and the (n,3n) reaction. In
these reactions there is assumed to be no correlation between the outgoing
energy and direction according to the ENDF/B-VII.1 manual 
\citep{chadwick_endf/b-vii.1_2011}. This simplification of the reaction model 
means that conservation of energy and momentum is unlikely for any given 
reaction. However the average of many individual reactions will still give the 
desired behavior. The neutron transfer probability for any of these reactions
can be represented as follows.
\begin{align}
  f(E^{'} \to E,\hat{\Omega}^{'} \to \hat{\Omega}) & = p(E^{'} \to E)
  p(\hat{\Omega}^{'} \to \hat{\Omega} | E^{'}) \nonumber \\
  & = p(E^{'} \to E)p(\hat{\Omega}^{'}\cdot\hat{\Omega} | E^{'}) \nonumber \\
  & = p(E^{'} \to E)p(\mu,E^{'})
\end{align}

As with elastic and inelastic level scattering the angular distributions are
expressed as normalized PDFs 
\citep{chadwick_endf/b-vii.1_2011}. 
\begin{equation*}
  p(\mu,E^{'}) = \frac{2\pi}{\sigma(E^{'})}\sigma(\mu,E^{'})
\end{equation*}
\begin{equation*}
  \int_{-1}^1p(\mu,E^{'})d\mu = 1
\end{equation*}
In addition the angular distrubition will be given in either a tabular form or
as a Legendre polynomial series. Usually, the outgoing angle cosine for these
reactions will be given in the lab frame \citep{chadwick_endf/b-vii.1_2011}. To
sample an outgoing angle cosine, the same tabular procedure that was outlined 
in the previous section should be used. 

The energy distributions are also expressed as normalized PDFs 
\citep{chadwick_endf/b-vii.1_2011}. The variable c is the number of neutrons
emitted from the reaction.
\begin{equation}
  p(E^{'} \to E) = \frac{1}{c\sigma(E^{'})}\frac{d\sigma(E^{'} \to E)}{dE}
\end{equation}
\begin{equation}
  \int_0^{E_{max}}p(E^{'} \to E)dE = 1
\end{equation}
The energy distributions are given in either a tabular form or as an analytical
formulation. The possible analytical formulations are the general evaporative
spectrum or the evaporation spectrum \citep{chadwick_endf/b-vii.1_2011}. The 
general evaporative spectrum has the following form.
\begin{equation}
  p(E^{'} \to E) = g(E/\theta(E^{'}))
\end{equation}
The function $\theta(E^{'})$ is tabulated as a function of incident neutron 
energy and the function $g(x)$ is tabulated as a function of 
$x = E/\theta(E^{'})$. To sample from this function a CDF corresponding to 
the pdf g(x) must be created. One then samples a random number and finds the
value of x that corresponds to the random number in the CDF that was 
calculated. Finally, the outgoing energy is calculated as follows.
\begin{equation}
  E = x\theta(E^{'})
\end{equation}

The evaporation spectrum, which is used for most non-fission reactions, has the following form \citep{chadwick_endf/b-vii.1_2011}.
\begin{equation}
  p(E^{'} \to E) = \frac{E}{I}exp\left[\frac{-E}{\theta(E^{'})}\right]
\end{equation}
\begin{equation}
  I = \theta^2(E^{'})\left[1 - exp\left[\frac{-(E^{'}-U)}{\theta(E^{'})}\right]
    \left(1+\frac{E^{'}-U}{\theta(E^{'})}\right)\right]
\end{equation}
The function $\theta(E^{'})$ is again tabulated as a function of incident
neutron energy and the variable U is introduced to define the proper upper 
limit for the final particle energy. Sampling from this distribution is also
quite simple. However, the derivation of the sampling procedure is slightly
complicated. Consider another PDF that has the same shape as the 
evaporation spectrum but extends to infinity (instead of $E^{'}-U$). 
\begin{equation}
  f(E^{'} \to E) = \frac{E}{\theta^2(E^{'})}exp\left[\frac{-E}{\theta(E^{'})}
    \right]
\end{equation}
Also assume that the energy E is the sum of two values $E_1$ and $E_2$, which
are independent of each other. If the PDF corresponding to $E_1$ is
\begin{equation}
  f_1(E^{'} \to E_1) = \frac{1}{\theta(E^{'})}exp\left[\frac{-E_1}{\theta(E^{'})}
    \right]
\end{equation}
and the PDF corresponding to $E_2$ is 
\begin{equation}
  f_2(E^{'} \to E_2) = \frac{1}{\theta(E^{'})}exp\left[\frac{-E_2}{\theta(E^{'})}
    \right]
\end{equation}
then the PDF for E is the following \citep{kahn_applications_1956}.
\begin{align}
  f(E^{'} \to E) & = \int_0^{E}f_1(E^{'} \to E-E_2)f_2(E^{'} \to E_2) dE_2 \\
  & = \frac{1}{\theta^2(E^{'})}\int_0^{E} exp\left[\frac{-(E-E_2)}{\theta(E^{'})}
    \right] exp\left[\frac{-E_2}{\theta(E^{'})}\right] dE_2 \nonumber \\
  & = \frac{1}{\theta^2(E^{'})} exp\left[\frac{-E}{\theta(E^{'})}\right]
  \int_0^{E} dE_2 \nonumber \\
  & = \frac{E}{\theta^2(E^{'})}exp\left[\frac{-E}{\theta(E^{'})}\right] \nonumber
\end{align}
Both $E_1$ and $E_2$ can be sampled from their respective PDFs using the
inverse CDF method. 
\begin{align}
  E_1 & = -\theta(E^{'})\ln{\varepsilon_1} \\
  E_2 & = -\theta(E^{'})\ln{\varepsilon_2} 
\end{align}
Since E is the sum of these two values, the equation for sampling E directly
is the following \citep{x-5_monte_carlo_team_mcnp_2003}.
\begin{align}
  E & = E_1 + E_2 \nonumber \\
  & = -\theta(E^{'})\ln{(\varepsilon_1\varepsilon_2)}
\end{align}
Finaly, since E was sampled from a distrubition that wasn't truncated properly,
the value of E must be rejected if it is greater than $E^{'}-U$ to account for
the improper truncation.

\section{Neutron Induced Fission}
The energy energy-dependent Watt spectrum
has the following form.
\begin{equation}
  p(E^{'} \to E) = \frac{1}{I}exp\left[\frac{-E}{a}\right]sinh\left(\sqrt{bE}
  \right)
\end{equation}
\begin{equation}
  I = \frac{1}{2}\sqrt{\frac{\pi a^3b}{4}}exp\left(\frac{ab}{4}\right)\left[
    erf\left(\sqrt{\frac{E^{'}-U}{a}} - \sqrt{\frac{ab}{4}}\right) +
    erf\left(\sqrt{\frac{E^{'}-U}{a}} + \sqrt{\frac{ab}{4}}\right)\right] -
    a exp\left[-\left(\frac{E^{'}-U}{a}\right)\right]sinh\sqrt{b(E^{'}-U)}
\end{equation}
The variables a and b are tabulated as a function of incident neutron energy 
and the variable U is introduced to again define the proper upper limit for the
final particle energy. 

\section{Thermal Scattering}

\section{Adjoint Elastic and Inelastic Level Scattering}

\section{Other Non-fission Adjoint  Reactions}

\section{Adjoint Neutron Induced Fission}

\section{Adjoint Thermal Scattering}


